% !TEX program = xelatex
\documentclass[10pt, a4paper, twocolumn]{article}

\usepackage{xeCJK}
\usepackage{enumerate}
\usepackage[colorlinks,
            linkcolor=blue,
            anchorcolor=blue,
            citecolor=green
            ]{hyperref}

%\setCJKmainfont{STFangsong}
%\setCJKmonofont{STFangsong}

\title{\textbf{JPAs·计算物理讨论班章程}}
\author{Liyang, Yidi7, Gaoxf}

\begin{document}

  \maketitle

  \section{讨论班简介}
%    \begin{description}
%      \item[时间] 2017年秋季学期
%      \item[地点] 唐敖庆楼B417
%      \item[课时] 计划从第二周起, 至考试周前结束, 每周一次
%      \item[面向群体] 对编程和计算机感兴趣的吉林大学本科生
%      \item[基本思想] 切身体验为主, 理论学习为辅. 在实践中学习, 在学习下实践.
%  \end{description}
   \textbf{时间}\\2017年秋季学期\\
   \textbf{地点}\\唐敖庆楼B417\\
   \textbf{课时}\\计划从第二周起, 至考试周前结束, 每周一次\\
   \textbf{面向群体}\\对编程和计算机感兴趣的吉林大学本科生\\
   \textbf{组织者\footnote{如有任何困难或问题, 请及时与三位组织者
          \href{https://github.com/JLUComPhy/JLU_Computational_Physics/%
              blob/master/Organizer_info.md}{联系}}}\\齐奕迪、李洋、高晓峰\\
   \textbf{基本思想}\\切身体验为主, 理论学习为辅. 在实践中学习, 在学习下实践.

  \section{讨论内容}
  本次计算物理讨论班内容包含计算物理基础的多个方面, \textbf{部分且粗略地}提出如下几点:
    \begin{enumerate}
      \item \textbf{基本技能}
      \begin{enumerate}[i)]
        \item \textbf{*nix}系统的使用
        \item 服务器的使用和维护
        \item \href{https://github.com/louisstuart96/lshort-new-zh-cn/%
                    blob/master/lshort-zh-cn.pdf}
                   {{\LaTeX}基础}
        \item gnuplot的使用
        \item \ldots
      \end{enumerate}  

      \item \textbf{专业知识}
      \begin{enumerate}[i)]
        \item 计算物理课内知识
        \item 凝聚态计算软件的使用
        \item 粒子加速器数据处理
        \item 机器学习与大数据处理
        \item \ldots
      \end{enumerate}

      \item \textbf{编程体验}
      \begin{enumerate}[i)]
        \item 相变的模拟
        \item 元胞自动机
        \item \href{https://github.com/JLUComPhy/JLU_Computational%
            _Physics/tree/master/Python/2nd_Lesson}{一个RPG游戏的兴起}
        \item \ldots
      \end{enumerate}

      \item \textbf{更多内容, 期待同学们的提出\ldots}
    \end{enumerate}

  \section{要求}
  基于吉林大学物理学讨论班的总章程, 2017年秋季学期计算物理讨论班另对参加的同学有如下要求:
  \begin{enumerate}
    \item 不论是出于美观方便还是以后学习工作需要的考虑, \emph{本期讨论班整理笔记必须使
          用{\LaTeX}}\footnote{由本文的排版, 各位或许可以感受到{\LaTeX}的强大之处.}.
          这也是我们打算, 将
          \href{https://github.com/louisstuart96/lshort-new-zh-cn/blob/%
                master/lshort-zh-cn.pdf}
          {``{\LaTeX}基础''}放在计算物理课程最前面学习讨论的原因.
    \item 请自备一台个人电脑, 无性能要求, 每次讨论班都需要带PC讨论.
    \item 每次活动签到, 用于期末评定奖励. 签到形式目前仍在研究当中.
    \item 讨论班开始前, 建议至少学习和灵活使用一种编辑器.
    (推荐: \href{http://www.vim.org/}{vim},
          \href{https://code.visualstudio.com}{vscode},
          \href{https://atom.io}{atom},
          \href{http://www.gnu.org/software/emacs/}{emacs}\ldots
          \footnote{下载使用前, 请注意看清各编辑器所支持的操作系统.})
    \item 建议至少提前掌握一种编程语言(如: C, C++, C\#, Java, Python, Matlab\ldots)
    \item 讨论班开始前, 建议熟悉git/\href{https://github.com}{GitHub}
          (强大的代码分享机制)的\href{https://git-scm.com/book/zh/v2}{使用}.
    \item 每次讨论结束后, 需将自己的笔记、感想或代码上传到\href{https://github.com}
          {GitHub}公共账号\footnote{相关账号将在第一次讨论班上公布.}上.
    \item 课下提问、讨论或答疑在\href{https://github.com/JLUComPhy/JLU_Computat%
                ional_Physics/blob/master/JPAs_Discussion.md}{GoogleGroups}中进行.
  \end{enumerate}
\end{document}